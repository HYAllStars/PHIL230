\documentclass{article}
\author{Hyeyoung Shin}
\title{Short Essay: HDT, MLK, and the Colin Ferguson Incident}
\date{22 Sept 2014}

\begin{document}
\maketitle

\section{Introduction}

Michael Brown has created tension between the police and protesters in
Ferguson. What really happened on the day of the fatal shooting remains
unclear. But the anger and unrest that followed the incident might have been
shared or supported by Thoreau and King to some extent.  

\subsection{Some Background on HDT}

Henry David Thoreau and Martin Luther King, jr. were supporters of civil rights
movements of their time. They both advocated a form of civil disorder against
unjust laws. In his essay, Civil Disobedience, Thoreau argues that government is
mere machine that we created for our own convenience, thus having obligation to
obey what it orders does not make sense. Thus, Thoreau would have viewed
Ferguson police as a standing army unjustly exercising authority on
individual's right to decide what's right and wrong and act accordingly.  

\subsection{Some Background on MLK}
Similarly, Martin Luther King, jr. affirmed that as much as we respect just
laws, one has a moral responsibility not to obey unjust laws. He would have
advised Lesley McSpadden, Brown's mother, to do what she did at a press
conference. Appealing for justice and no violence.(USA TODAY) 

Then what kind of justice Thoreau and King would have sought in Ferguson?

government is only a tool as easily abused and perverted as a standing
army. Thus the physically strong majority violates individual's rights to do
what he/she thinks is right through law enforcement.  

. would see it as wounded individual's right by government. King would see it as
racial injustice.  

\section{HDT and Ferguson}

A 19 century abolitionist Henry David Thoreau viewed government and its law
enforcement unnecessary evil created by ourselves. He advocated civil
disobedience in his book arguing that as much as we object a standing army we
should be against government as the standing army is only an arm of the standing
government.(Civil Disobedience) Thus, Thoreau might have seen Ferguson police as
the physical force unjustly violating people's right to decide what's right and
wrong and act accordingly.  

Similarly, Martin Luther King, jr. affirmed that as much as we respect just
laws, one has a moral responsibility not to obey unjust laws. He would have
advised Lesley McSpadden, Brown's mother, to do what she did at a press
conference. Appealing for justice and no violence.(USA TODAY) 

Then what kind of justice Thoreau and King would have sought in Ferguson?

Thoreau would not have seen the police officer as an agent of injustice nor his
action serving the state if the officer believed all he was doing was what he
thinks is right. In fact Brown was involved in the armed robbery reported. But
even if the alleged physical assaults and a struggle over the officer's gun
really happened(CNN.com), it would appear to be unjust to Thoreau that the
police carries a gun in the first place. Thoreau would say it is absurd that
people pay taxes to buy more guns for the law enforcement however justly used or
not. The kind of protection government provides with armed forces would be
blatantly rejected by him.     

\section{King and Ferguson}

King would see racial injustice in the center of the problem. By actively
engaged in the matter encouraging peaceful protests and negotiating to change
unjust laws, he would try to prove that a single gain in civil rights cannot be
made without determined legal and nonviolent pressure.(Letter from a Birmingham
Jail) From what he might have collected facts to determine whether injustice
existed in the investigation of the police officer who killed Brown. Based on
the facts collected from various sources, the officer was put on paid leave
after the shooting and his name and the details of the shooting were not
announced 

\section{HDT and MLK}

\section{Conclusion}

\end{document}
